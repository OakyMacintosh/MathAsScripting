% Book wrote by Miguel Vicente Mesquita/SufremOak
% i wrote this whole book using the overleaf latex guide

\documentclass{book}
\usepackage{graphicx}
\usepackage{amsmath}
\graphicspath{{images/}} % Configuring the graphicx package
\title{The MathAsScripting Book}
\author{Miguel Vicente Mesquita}
\date{April 2024}
\begin{document}
\maketitle
The MathAsScripting Book
Made in \LaTeX{}

\newpage

\chapter{First Chapter}

\section{Introduction}

\paragraph{What it is?}
MathAsScripting is a solo project made by Miguel V. Mesquita to make math more
like a programming language, like python or crystal in the DeclMath case

\paragraph{But why?}
``I made it to add more sense to math, because all the formulas and other things don't get into my mind''

\paragraph{Understanding the concepts}
All this MathAsScripting project envolves itself at a unic mathematical function:
\{ \sum \longleftrightarrow \varDelta² \} 
This function isn't a common math function, it is a Declarative-Mathetical-Oriented-Function, or DMOF,
DMOF functions are constant based operations that changes a variable contantly in a period of ticks,
which can be metered in computer bits (sequences of 0s and 1s, or binary), in a 8-bit format.

DMOF commonly uses simuntaneous symbols like:
\Longleftrightarrow \longleftrightarrow \leftrightarrow \Leftrightarrow \Updownarrow \updownarrow \nleftrightarrow \nLeftrightarrow 
which can create and change constants

by doing
\mathbf{A} \neq \sqrt{200} \div \tan{6}
you will get
\mathbf{A} \eq \exp \leftrightsquigarrow \mathcal{E}
Which is
\mathbf{A} \eq \mathbf{10} 

Confusing? Of course, but it is necessary to understand DeclMath.

\chapter{Getting started}
    \section{Starting with DeclMath}
    DeclMath is an math-based language (obviously) and also an Declarative language,
    as with its looks, it is an DSL language for math purposes, with scientific features
    and some phisics features, and also Mind-based (can "run" on your mind).
    \paragraph{}
    Starting with it is a really simple task, and if you want to make a "hello world"
    with it, good luck, because it is a bit harder than you can think, that language isn't
    made for printing common ASCII characters like Aa-Zz, it is only expression powered,
    which means you can only do mathematical operations and some other things, but a hello world
    isn't a easy task without using any generators, like the Markin Programming language generator
    script.
    \paragraph{}
    To make a "hello, world" operand you will need to get an binary value for it
    so:

    Letter H:
    \mathbf{H} \eq \mathbf{7.2 \times 10^1}
    Letter E:
    \mathbf{E} \eq \mathbf{1.000101 \times 10^8}
    Letter L:
    \mathbf{L} \eq \mathbf{1.0011 \times 10^6}
    Letter O:
    \mathbf{O} \eq \mathbf{7.2 \times 10^1^2}

    here's an implementation in DeclMath:

    \begin{lstlisting}
        #lang 'en'

        # map the ascii characters
        h = expr(7.2&.&10^1);
        e = expr(1.00010&.&10^8)
        l = expr(1.0011&.&10^6)
        o = expr(7.2&.&10^1^2)

        f(get h, e, l, o):
            for \n in 5:
                show(h)
                show(e)
                show(l) in 2^1
                show(o) 
    \end{lstlisting}

    that will output:
        \begin{lstlisting}
            hello
        \end{lstlisting}

    you probably noticed those strange expressions:
    \begin{lstlisting}
        h = expr(7.2&.&10^1);
        e = expr(1.00010&.&10^8)
        l = expr(1.0011&.&10^6)
        o = expr(7.2&.&10^1^2)
    \end{lstlisting}

    those are the binary code of the ASCII letters "H", "E", "L" and "O"
    in scientific notation, which is the only method of binary input in DeclMath.

    

\end{document}